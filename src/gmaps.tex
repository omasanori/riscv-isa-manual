\chapter{RV32/64G Instruction Set Listings}

One goal of the RISC-V project is that it be used as a stable software
development target.  For this purpose, we define a combination of a
base ISA (RV32I or RV64I) plus selected standard extensions (IMAFD, Zicsr, Zifencei) as
a ``general-purpose'' ISA, and we use the abbreviation G for the IMAFDZicsr\_Zifencei
combination of instruction-set extensions.    This chapter presents
opcode maps and instruction-set listings for RV32G and RV64G.

\vspace{0.1in}
\definecolor{gray}{RGB}{180,180,180}
\begin{table*}[htbp]
\begin{center}
{\footnotesize
\setlength{\tabcolsep}{4pt}
\begin{tabular}{|r|c|c|c|c|c|c|c|c|}
  \hline
  inst[4:2] & 000    & 001      & 010            & 011      & 100    & 101            & 110                  & 111 \\ \cline{1-1}
  inst[6:5] &        &          &                &          &        &                &                      &   \\ \hline
         00 & LOAD   & LOAD-FP  & {\em custom-0} & MISC-MEM & OP-IMM & AUIPC          & OP-IMM-32            & {\em reserved}  \\ \hline
         01 & STORE  & STORE-FP & {\em custom-1} & AMO      & OP     & LUI            & OP-32                & {\em reserved}  \\ \hline
         10 & MADD   & MSUB     & NMSUB          & NMADD    & OP-FP  & {\em reserved} & {\em custom-2}       & {\em reserved}  \\ \hline
         11 & BRANCH & JALR     & {\em reserved} & JAL      & SYSTEM & {\em reserved} & {\em custom-3}       & {\em reserved}  \\ \hline

 \end{tabular}
}
\end{center}
\vspace{-0.15in}
\caption{RISC-V base opcode map, inst[1:0]=11}
\label{opcodemap}
\end{table*}


Table~\ref{opcodemap} shows a map of the major opcodes for RVG.  Major
opcodes with 3 or more lower bits set are reserved for instruction
lengths greater than 32 bits.  Opcodes marked as {\em reserved} should
be avoided for custom instruction-set extensions as they might be used
by future standard extensions.  Major opcodes marked as {\em custom-0}
and {\em custom-1} will be avoided by future standard extensions and
are recommended for use by custom instruction-set extensions within
the base 32-bit instruction format.

We believe RV32G and RV64G provide simple but complete instruction
sets for a broad range of general-purpose computing.  The optional
compressed instruction set described in Chapter~\ref{compressed} can
be added (forming RV32GC and RV64GC) to improve performance, code
size, and energy efficiency, though with some additional hardware
complexity.

As we move beyond IMAFDC into further instruction-set extensions, the
added instructions tend to be more domain-specific and only provide
benefits to a restricted class of applications, e.g., for multimedia
or security.  Unlike most commercial ISAs, the RISC-V ISA design
clearly separates the base ISA and broadly applicable standard
extensions from these more specialized additions.

\input{instr-table}

\FloatBarrier
Table~\ref{rvgcsrnames} lists the CSRs that have
currently been allocated CSR addresses.  The timers, counters, and
floating-point CSRs are the only CSRs defined in this specification.

\begin{table}[htb!]
\begin{center}
\begin{tabular}{|l|l|l|l|}
\hline
Number    & Privilege & Name & Description \\
\hline
\multicolumn{4}{|c|}{Floating-Point Control and Status Registers} \\
\hline
\tt 0x001 & Read/write  &\tt fflags     & Floating-Point Accrued Exceptions. \\
\tt 0x002 & Read/write  &\tt frm        & Floating-Point Dynamic Rounding Mode. \\
\tt 0x003 & Read/write  &\tt fcsr       & Floating-Point Control and Status
Register ({\tt frm} + {\tt fflags}). \\
\hline
\multicolumn{4}{|c|}{Counters and Timers} \\
\hline
\tt 0xC00 & Read-only  &\tt cycle      & Cycle counter for RDCYCLE instruction. \\
\tt 0xC01 & Read-only  &\tt time       & Timer for RDTIME instruction. \\
\tt 0xC02 & Read-only  &\tt instret    & Instructions-retired counter for RDINSTRET instruction. \\
\tt 0xC80 & Read-only  &\tt cycleh     & Upper 32 bits of {\tt cycle}, RV32I only. \\
\tt 0xC81 & Read-only  &\tt timeh      & Upper 32 bits of {\tt time}, RV32I only. \\
\tt 0xC82 & Read-only  &\tt instreth   & Upper 32 bits of {\tt instret}, RV32I only. \\
\hline
\end{tabular}
\end{center}
\caption{RISC-V control and status register (CSR) address map.}
\label{rvgcsrnames}
\end{table}
